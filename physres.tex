%% Version 4.3.2, 25 August 2014
%
%%%%%%%%%%%%%%%%%%%%%%%%%%%%%%%%%%%%%%%%%%%%%%%%%%%%%%%%%%%%%%%%%%%%%%
% Template.tex --  LaTeX-based template for submissions to the 
% American Meteorological Society
%
% Template developed by Amy Hendrickson, 2013, TeXnology Inc., 
% amyh@texnology.com, http://www.texnology.com
% following earlier work by Brian Papa, American Meteorological Society
%
% Email questions to latex@ametsoc.org.
%
%%%%%%%%%%%%%%%%%%%%%%%%%%%%%%%%%%%%%%%%%%%%%%%%%%%%%%%%%%%%%%%%%%%%%
% PREAMBLE
%%%%%%%%%%%%%%%%%%%%%%%%%%%%%%%%%%%%%%%%%%%%%%%%%%%%%%%%%%%%%%%%%%%%%

%% Start with one of the following:
% DOUBLE-SPACED VERSION FOR SUBMISSION TO THE AMS
%\documentclass{ametsoc}

% TWO-COLUMN JOURNAL PAGE LAYOUT---FOR AUTHOR USE ONLY
 \documentclass[twocol]{ametsoc}

%%%%%%%%%%%%%%%%%%%%%%%%%%%%%%%%
%%% To be entered only if twocol option is used

\journal{mwr}

%  Please choose a journal abbreviation to use above from the following list:
% 
%   jamc     (Journal of Applied Meteorology and Climatology)
%   jtech     (Journal of Atmospheric and Oceanic Technology)
%   jhm      (Journal of Hydrometeorology)
%   jpo     (Journal of Physical Oceanography)
%   jas      (Journal of Atmospheric Sciences)	
%   jcli      (Journal of Climate)
%   mwr      (Monthly Weather Review)
%   wcas      (Weather, Climate, and Society)
%   waf       (Weather and Forecasting)
%   bams (Bulletin of the American Meteorological Society)
%   ei    (Earth Interactions)

%%%%%%%%%%%%%%%%%%%%%%%%%%%%%%%%
%Citations should be of the form ``author year''  not ``author, year''
\bibpunct{(}{)}{;}{a}{}{,}

%%%%%%%%%%%%%%%%%%%%%%%%%%%%%%%%

%%% To be entered by author:

%% May use \\ to break lines in title:

\title{Physics-dynamics coupling with element-based high-order Galerkin methods: quasi equal-area physics grid}

%%% Enter authors' names, as you see in this example:
%%% Use \correspondingauthor{} and \thanks{Current Affiliation:...}
%%% immediately following the appropriate author.
%%%
%%% Note that the \correspondingauthor{} command is NECESSARY.
%%% The \thanks{} commands are OPTIONAL.

    %\authors{Author One\correspondingauthor{Author One, 
    % American Meteorological Society, 
    % 45 Beacon St., Boston, MA 02108.}
% and Author Two\thanks{Current affiliation: American Meteorological Society, 
    % 45 Beacon St., Boston, MA 02108.}}

\authors{Adam R. Herrington\correspondingauthor{Peter H. Lauritzen, Climate and Global Dynamics, National Center for Atmospheric Research, 1850 Table Mesa Drive, Boulder, Colorado, USA.}}

%% Follow this form:
    % \affiliation{American Meteorological Society, 
    % Boston, Massachusetts.}

\affiliation{School of Marine and Atmospheric Sciences, Stony Brook University, State University of New York, Stony Brook, New York.}

%% Follow this form:
    %\email{latex@ametsoc.org}

\email{pel@ucar.edu}

%% If appropriate, add additional authors, different affiliations:
\extraauthor{Peter H. Lauritzen}
\extraaffil{Climate and Global Dynamics, National Center for Atmospheric Research, 1850 Table Mesa Drive, Boulder, Colorado, USA.}
\extraauthor{Steve Goldhaber}
\extraaffil{Climate and Global Dynamics, National Center for Atmospheric Research, 1850 Table Mesa Drive, Boulder, Colorado, USA.}
\extraauthor{Kevin A. Reed}
\extraaffil{School of Marine and Atmospheric Sciences, Stony Brook University, State University of New York, Stony Brook, New York.}

%% May repeat for a additional authors/affiliations:

%\extraauthor{}
%\extraaffil{}

%%%%%%%%%%%%%%%%%%%%%%%%%%%%%%%%%%%%%%%%%%%%%%%%%%%%%%%%%%%%%%%%%%%%%
% ABSTRACT
%
% Enter your abstract here
% Abstracts should not exceed 250 words in length!
%
% For BAMS authors only: If your article requires a Capsule Summary, please place the capsule text at the end of your abstract
% and identify it as the capsule. Example: This is the end of the abstract. (Capsule Summary) This is the capsule summary. 

\abstract{Enter the text of your abstract here.}

\begin{document}

%% Necessary!
\maketitle


%%%%%%%%%%%%%%%%%%%%%%%%%%%%%%%%%%%%%%%%%%%%%%%%%%%%%%%%%%%%%%%%%%%%%
% MAIN BODY OF PAPER
%%%%%%%%%%%%%%%%%%%%%%%%%%%%%%%%%%%%%%%%%%%%%%%%%%%%%%%%%%%%%%%%%%%%%
%

%% In all cases, if there is only one entry of this type within
%% the higher level heading, use the star form: 
%%
\begin{itemize}
\item We should compute TKE of tendencies and show that we ar removing scales with nc<2
\end{itemize}
\section{Introduction}

%this should be included in the next paper (physres)
%When introducing a physics grid separate from the dynamics grid the question arises of what the resolution of the physics grid should be compared to the dynamics grid. For example, Figure \ref{fig:physgrid-1d} shows physics grids with the same, coarser and finer resolution than the GLL dynamics grid. From linear stability and accuracy analysis of numerical methods, it is a common result that the shortest resolvable wavelengths are not accurately represented. Similar arguments can be made from analyzing total kinetic energy spectra \citep{S2011LNCSE}. One may therefore argue that only believable scales should be passed to the physical parameterizations \citep{LH1997MWR}, i.e. a coarser resolution physics grid. This concept was investigated in a spectral transform model by \cite{W1999T}. On the other hand, computing physics tendencies on a higher-resolution grid compared to the dynamical core may provide a better sampling of the atmospheric state, somewhat similar to the super-parameterization \citep{G2001JAS,GRL:GRL14999,SA2007ASL} and sub-columns{\footnote{\citet{gmdd-8-5041-2015} in the context of CAM}} \citep{subcolumn,JGRD:JGRD10481} concepts. This approach was taken by \cite{M2009T} in the context of vertical refinement. \cite{W2014PTRSL} found improved forecast scores by increasing the grid-point space resolution compared to the resolution in wave-number space for the spectral transform model at ECMWF. 

%Alternatively, one could combine the two ideas and compute the state of the atmosphere on a coarser resolution grid and then use sub-columns or super-parameterization. One thereby passes believable scales to the sub-grid scale model and thereby assumes that a statistical sampling, in the case of sub-columns, or a simplified cloud-resolving model, in the case of super-parameterization, provides a more accurate sub-grid-scale tendency than sampling the Galerkin basis functions over the sub-grid-scale. [Discuss \cite{W2014PTRSL}: spectral truncation and physics (physical) grid (see page 10; conclusions)]




%Separating physics and dynamics grids has been investigated in the context of spectral transform models by \citet{TELA:TELA0009}, in which the separation was performed by truncation in wavenumber space. Parts of the physical parameterizations (microphysics) were separated in \citet{JGRD:JGRD50711} and vertical grid separation was investigated in \citet{TELA:TELA394}. In this study we run all of the parameterization on the physgrid.


%In the context of spectral transform model \citet{TELA:TELA0009} held the physics forcing scale fixed while refining the horizontal resolution of the dynamical core. In the context of microphysics \citet{JGRD:JGRD50711} did a scale separation and \citet{TELA:TELA394} separated the vertical grids. While \citet{LH1997MWR} and \citet{TELA:TELA0009} separated scales truncation of the the wave transform, we here separate scales through integrating basis functions over control volumes and , contrary to \citet{JGRD:JGRD50711}, all of the physical parameterization computations are performed on the physgrid. We focus on horizontal separation of scales here.


% \subsection*{subsection}
% text...
\section{Methods}

% \section{Section title}
% \subsection{subsection one}
% text...
% \subsection{subsection two}
% \section{Section title}

%%%

\section{Results}

% \subsection{First secondary heading}

% \subsubsection{First tertiary heading}

% \paragraph{First quaternary heading}

%%%%%%%%%%%%%%%%%%%%%%%%%%%%%%%%%%%%%%%%%%%%%%%%%%%%%%%%%%%%%%%%%%%%%
% ACKNOWLEDGMENTS
%%%%%%%%%%%%%%%%%%%%%%%%%%%%%%%%%%%%%%%%%%%%%%%%%%%%%%%%%%%%%%%%%%%%%
%
\acknowledgments
NCAR is sponsored by the National Science Foundation (NSF).

%%%%%%%%%%%%%%%%%%%%%%%%%%%%%%%%%%%%%%%%%%%%%%%%%%%%%%%%%%%%%%%%%%%%%
% APPENDIXES
%%%%%%%%%%%%%%%%%%%%%%%%%%%%%%%%%%%%%%%%%%%%%%%%%%%%%%%%%%%%%%%%%%%%%
%
% Use \appendix if there is only one appendix.
%\appendix

% Use \appendix[A], \appendix}[B], if you have multiple appendixes.
%\appendix[A]

%% Appendix title is necessary! For appendix title:
%\appendixtitle{}

%%% Appendix section numbering (note, skip \section and begin with \subsection)
% \subsection{First primary heading}

% \subsubsection{First secondary heading}

% \paragraph{First tertiary heading}

%% Important!
%\appendcaption{<appendix letter and number>}{<caption>} 
%must be used for figures and tables in appendixes, e.g.,
%
%\begin{figure}
%\noindent\includegraphics[width=19pc,angle=0]{figure01.pdf}\\
%\appendcaption{A1}{Caption here.}
%\end{figure}
%
% All appendix figures/tables should be placed in order AFTER the main figures/tables, i.e., tables, appendix tables, figures, appendix figures.
%
%%%%%%%%%%%%%%%%%%%%%%%%%%%%%%%%%%%%%%%%%%%%%%%%%%%%%%%%%%%%%%%%%%%%%
% REFERENCES
%%%%%%%%%%%%%%%%%%%%%%%%%%%%%%%%%%%%%%%%%%%%%%%%%%%%%%%%%%%%%%%%%%%%%
% Make your BibTeX bibliography by using these commands:
\bibliographystyle{ametsoc2014}
\bibliography{bib}


%%%%%%%%%%%%%%%%%%%%%%%%%%%%%%%%%%%%%%%%%%%%%%%%%%%%%%%%%%%%%%%%%%%%%
% TABLES
%%%%%%%%%%%%%%%%%%%%%%%%%%%%%%%%%%%%%%%%%%%%%%%%%%%%%%%%%%%%%%%%%%%%%
%% Enter tables at the end of the document, before figures.
%%
%
%\begin{table}[t]
%\caption{This is a sample table caption and table layout.  Enter as many tables as
%  necessary at the end of your manuscript. Table from Lorenz (1963).}\label{t1}
%\begin{center}
%\begin{tabular}{ccccrrcrc}
%\hline\hline
%$N$ & $X$ & $Y$ & $Z$\\
%\hline
% 0000 & 0000 & 0010 & 0000 \\
% 0005 & 0004 & 0012 & 0000 \\
% 0010 & 0009 & 0020 & 0000 \\
% 0015 & 0016 & 0036 & 0002 \\
% 0020 & 0030 & 0066 & 0007 \\
% 0025 & 0054 & 0115 & 0024 \\
%\hline
%\end{tabular}
%\end{center}
%\end{table}

%%%%%%%%%%%%%%%%%%%%%%%%%%%%%%%%%%%%%%%%%%%%%%%%%%%%%%%%%%%%%%%%%%%%%
% FIGURES
%%%%%%%%%%%%%%%%%%%%%%%%%%%%%%%%%%%%%%%%%%%%%%%%%%%%%%%%%%%%%%%%%%%%%
%% Enter figures at the end of the document, after tables.
%%
%
%\begin{figure}[t]
%  \noindent\includegraphics[width=19pc,angle=0]{figure01.pdf}\\
%  \caption{Enter the caption for your figure here.  Repeat as
%  necessary for each of your figures. Figure from \protect\cite{Knutti2008}.}\label{f1}
%\end{figure}

\end{document}
