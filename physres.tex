%%%%%%%%%%%%%%%%%%%%%%%%%%%%%%%%%%%%%%%%%%%%%%%%%%%%%%%%%%%%%%%%%%%%%%%%%%%%
% AGUJournalTemplate.tex: this template file is for articles formatted with LaTeX
%
% This file includes commands and instructions
% given in the order necessary to produce a final output that will
% satisfy AGU requirements. 
%
% You may copy this file and give it your
% article name, and enter your text.
%
%%%%%%%%%%%%%%%%%%%%%%%%%%%%%%%%%%%%%%%%%%%%%%%%%%%%%%%%%%%%%%%%%%%%%%%%%%%%
% PLEASE DO NOT USE YOUR OWN MACROS
% DO NOT USE \newcommand, \renewcommand, or \def, etc.
%
% FOR FIGURES, DO NOT USE \psfrag or \subfigure.
% DO NOT USE \psfrag or \subfigure commands.
%%%%%%%%%%%%%%%%%%%%%%%%%%%%%%%%%%%%%%%%%%%%%%%%%%%%%%%%%%%%%%%%%%%%%%%%%%%%
%
% All questions should be e-mailed to latex@agu.org.
%
%%%%%%%%%%%%%%%%%%%%%%%%%%%%%%%%%%%%%%%%%%%%%%%%%%%%%%%%%%%%%%%%%%%%%%%%%%%%
%
% Step 1: Set the \documentclass
%
% There are two options for article format:
%
% 1) PLEASE USE THE DRAFT OPTION TO SUBMIT YOUR PAPERS.
% The draft option produces double spaced output.
% 
% 2) numberline will give you line numbers.

%% To submit your paper:
%\documentclass[draft,linenumbers]{agujournal}
%\draftfalse

%% For final version.
\documentclass{agujournal}
\usepackage{url}
% Now, type in the journal name: \journalname{<Journal Name>}

% ie, \journalname{Journal of Geophysical Research}
%% Choose from this list of Journals:
%
% JGR-Atmospheres
% JGR-Biogeosciences
% JGR-Earth Surface
% JGR-Oceans
% JGR-Planets
% JGR-Solid Earth
% JGR-Space Physics
% Global Biochemical Cycles
% Geophysical Research Letters
% Paleoceanography
% Radio Science
% Reviews of Geophysics
% Tectonics
% Space Weather
% Water Resource Research
% Geochemistry, Geophysics, Geosystems
% Journal of Advances in Modeling Earth Systems (JAMES)
% Earth's Future
% Earth and Space Science
%
%

\journalname{Journal of Advances in Modeling Earth Systems (JAMES)}


\begin{document}

%% ------------------------------------------------------------------------ %%
%  Title
% 
% (A title should be specific, informative, and brief. Use
% abbreviations only if they are defined in the abstract. Titles that
% start with general keywords then specific terms are optimized in
% searches)
%
%% ------------------------------------------------------------------------ %%

% Example: \title{This is a test title}

\title{Evaluating the physical parameterizations at a lower resolution in the Community Atmosphere Model with spectral-element dynamics}

%% ------------------------------------------------------------------------ %%
%
%  AUTHORS AND AFFILIATIONS
%
%% ------------------------------------------------------------------------ %%

% Authors are individuals who have significantly contributed to the
% research and preparation of the article. Group authors are allowed, if
% each author in the group is separately identified in an appendix.)

% List authors by first name or initial followed by last name and
% separated by commas. Use \affil{} to number affiliations, and
% \thanks{} for author notes.  
% Additional author notes should be indicated with \thanks{} (for
% example, for current addresses). 

% Example: \authors{A. B. Author\affil{1}\thanks{Current address, Antartica}, B. C. Author\affil{2,3}, and D. E.
% Author\affil{3,4}\thanks{Also funded by Monsanto.}}

\authors{A.R. Herrington\affil{1}\thanks{Stony Brook, New York}, P.H. Lauritzen\affil{2}, S. Goldhaber\affil{1}, Mark A. Taylor\affil{3}, K.A. Reed\affil{2}}

 \affiliation{1}{School of Marine and Atmospheric Sciences, Stony Brook University, Stony Brook, New York}
 \affiliation{2}{National Center for Atmospheric Research, Boulder, Colorado, USA}
 \affiliation{3}{Sandia National Laboratories, Albuquerque, New Mexico, USA}

% \affiliation{3}{Third Affiliation}
% \affiliation{4}{Fourth Affiliation}

%(repeat as many times as is necessary)

%% Corresponding Author:
% Corresponding author mailing address and e-mail address:

% (include name and email addresses of the corresponding author.  More
% than one corresponding author is allowed in this LaTeX file and for
% publication; but only one corresponding author is allowed in our
% editorial system.)  

% Example: \correspondingauthor{First and Last Name}{email@address.edu}

\correspondingauthor{Adam R. Herrington}{adam.herrington@stonybrook.edu}

%% Keypoints, final entry on title page.

% Example: 
% \begin{keypoints}
% \item	List up to three key points (at least one is required)
% \item	Key Points summarize the main points and conclusions of the article
% \item	Each must be 100 characters or less with no special characters or punctuation 
% \end{keypoints}

%  List up to three key points (at least one is required)
%  Key Points summarize the main points and conclusions of the article
%  Each must be 100 characters or less with no special characters or punctuation 

\begin{keypoints}
\item 
\item 
\item 
\end{keypoints}

%% ------------------------------------------------------------------------ %%
%
%  ABSTRACT
%
% A good abstract will begin with a short description of the problem
% being addressed, briefly describe the new data or analyses, then
% briefly states the main conclusion(s) and how they are supported and
% uncertainties. 
%% ------------------------------------------------------------------------ %%

%% \begin{abstract} starts the second page 

\begin{abstract}
\end{abstract}


%% ------------------------------------------------------------------------ %%
%
%  TEXT
%
%% ------------------------------------------------------------------------ %%

%%% Suggested section heads:
%\section{Introduction}
% 
% The main text should start with an introduction. Except for short
% manuscripts (such as comments and replies), the text should be divided
% into sections, each with its own heading. 

% Headings should be sentence fragments and do not begin with a
% lowercase letter or number. Examples of good headings are:

% \section{Materials and Methods}
% Here is text on Materials and Methods.
%
% \subsection{A descriptive heading about methods}
% More about Methods.
% 
% \section{Data} (Or section title might be a descriptive heading about data)
% 
% \section{Results} (Or section title might be a descriptive heading about the
% results)
% 
% \section{Conclusions}

%this should be included in the next paper (physres)
%When introducing a physics grid separate from the dynamics grid the question arises of what the resolution of the physics grid should be compared to the dynamics grid. For example, Figure \ref{fig:physgrid-1d} shows physics grids with the same, coarser and finer resolution than the GLL dynamics grid. From linear stability and accuracy analysis of numerical methods, it is a common result that the shortest resolvable wavelengths are not accurately represented. Similar arguments can be made from analyzing total kinetic energy spectra \citep{S2011LNCSE}. One may therefore argue that only believable scales should be passed to the physical parameterizations \citep{LH1997MWR}, i.e. a coarser resolution physics grid. This concept was investigated in a spectral transform model by \cite{W1999T}. On the other hand, computing physics tendencies on a higher-resolution grid compared to the dynamical core may provide a better sampling of the atmospheric state, somewhat similar to the super-parameterization \citep{G2001JAS,GRL:GRL14999,SA2007ASL} and sub-columns{\footnote{\citet{gmdd-8-5041-2015} in the context of CAM}} \citep{subcolumn,JGRD:JGRD10481} concepts. This approach was taken by \cite{M2009T} in the context of vertical refinement. \cite{W2014PTRSL} found improved forecast scores by increasing the grid-point space resolution compared to the resolution in wave-number space for the spectral transform model at ECMWF. 

%Alternatively, one could combine the two ideas and compute the state of the atmosphere on a coarser resolution grid and then use sub-columns or super-parameterization. One thereby passes believable scales to the sub-grid scale model and thereby assumes that a statistical sampling, in the case of sub-columns, or a simplified cloud-resolving model, in the case of super-parameterization, provides a more accurate sub-grid-scale tendency than sampling the Galerkin basis functions over the sub-grid-scale. [Discuss \cite{W2014PTRSL}: spectral truncation and physics (physical) grid (see page 10; conclusions)]

%Separating physics and dynamics grids has been investigated in the context of spectral transform models by \citet{TELA:TELA0009}, in which the separation was performed by truncation in wavenumber space. Parts of the physical parameterizations (microphysics) were separated in \citet{JGRD:JGRD50711} and vertical grid separation was investigated in \citet{TELA:TELA394}. In this study we run all of the parameterization on the physgrid.

%In the context of spectral transform model \citet{TELA:TELA0009} held the physics forcing scale fixed while refining the horizontal resolution of the dynamical core. In the context of microphysics \citet{JGRD:JGRD50711} did a scale separation and \citet{TELA:TELA394} separated the vertical grids. While \citet{LH1997MWR} and \citet{TELA:TELA0009} separated scales truncation of the the wave transform, we here separate scales through integrating basis functions over control volumes and , contrary to \citet{JGRD:JGRD50711}, all of the physical parameterization computations are performed on the physgrid. We focus on horizontal separation of scales here.

\section{Introduction}

Global atmospheric models fundamentally consist of two components. The dynamical core, which advances the equations of motion on a discrete grid, and the physical parameterizations, which compute the effect of diabatic and subgrid-scale processes (e.g., radiative transfer and moist convection) on the resolved scales. Conventionally, the physical paramterizations are evaluated on the dynamical core grid. From linear stability and accuracy analysis of numerical methods, it is a common result that the shortest resolvable wavelengths are not accurately represented by the dynamical core. Similar arguments can be made from analyzing total kinetic energy spectra \citep{S2011LNCSE}. One may therefore argue that only believable scales should be passed to the physical parameterizations \citep{LH1997MWR}, i.e. the physical parameterizations should be evaluated on a coarser grid. This concept was first investigated in a spectral transform model by \cite{W1999T}. 

In order to understand the results of \cite{W1999T}, it is necessary to understand the impact of grid resolution on the solution.  While no complete theory for the resolution sensitivty of global atmopsheric models exists, their are no shortage of convergence studies. These studies are often performed at resolutions typical of present day General Circulation Models (GCMs), and often use idealized aqua-planet boundary conditions. Aqua-planet configurations are useful for convergence studies since they are devoid of boundary conditions that would otherwise vary with resolution in more realistic configurations (e.g. topography), while still maintaing some resemblance of Earth's climate. A robust result from these convergence studies is that the strength of the Hadley Cell tends to increase with resolution (see \cite{HR2017JCLIM} and references therein), with some exceptions (see XXXX).

Likewise, \cite{W1999T} found the strength of Hadley Cell increased with resolution in their convergence study. But \cite{W1999T} then performed an additional test, one in which the dynamical core resolution was increased, but the physical paramterizations were evaluated from a truncated state, and that truncation wavenumber was held fixed across resolutions. The resulting Hadley Cell showed very little sensitivity to dynamical core resolution, resembling the solution for which the resolution of the dynamical core was equal to the resolution of the truncated physics forcing. \cite{W2007JMSJ} reflects on this study years later, suggesting that the scales of motion resolved by the dynamical core appeared to be aliased to the truncation scale of the physics forcing. This finding lead \cite{W2007JMSJ} to question the advantage of using a coarser physics grid to filter out the under-resolved scales, as advocated by \citep{LH1997MWR}. 

It is well known that the equations of motion have implicit scale-dependencies at hydrostatic scales \citep{O1981JAS}. Perhaps the most dramatic scale depedency occurs under gravitational instability, as the vertical velocity scale is proportional to the horizontal scale of the buoyancy. \cite{HR2017JCLIM} hypothesize that an increase in resolution leads to a proportional reduction in the horizontal scale of the physics forcing, resulting in larger magnitude vertical motion in the ascending branch of the Hadley Cell, and ultimately equilibrating with a stronger overturning circulation. This hypothesis is consistent with the results of \cite{W1999T}, since if the truncation scale of the physics forcing is held fixed, the vertical motion will be aliased to horizontal scale of the coarser physics forcing, and the vertical velocity scale will be unchanged.


\section{Methods}
Separating dynamics, tracer and physics grids introduces the added complexity of having to map the state from dynamics and tracer grids to the physics grid; and mapping physics tracer tendencies back to the tracer grid and physics tendencies needed by the dynamical core to the dynamics grid. The dynamics grid refers to the Gauss-Lobatto-Legendre (GLL) quadrature nodes used by the spectral-element method to solve the momentum equations for the momentum vector $(u,v)$, thermodynamics equation for temperature ($T$), continuity equation for dry air ($p$), and continuity equations for water vapor and condensates thermodynamically active \citep[see, e.g., ][ for details]{LetAl2018JAMES}. By tracer grid we refer to the $pg3$ grid on which CSLAM performs tracer transport of water vapor, condensates and other tracers. The GLL value for water vapor and condensates is overwritten by the CSLAM values every physics time-step so that the spectral-element advection of water species does not become decoupled from the the CSLAM advection of the same water species. Mapping velocity components, dry air mass and temperature from the GLL grid to the $pg2$ grid is done by using the internal degree 3 Lagrange basis functions in CAM-SE \citep[as described in  ][ for pg3; exactly the same methods can be used for $pg2$]{HL2018MWR}.

As compared to the $pg3$ configuration, the extra complication of the $pg2$ setup is that tracer state needs to be mapped from the tracer grid to the physics grid and tracer tendencies need to the mapped from the physics grid to CSLAM grid. In order to describe the algorithm some notation needs to be introduced.

The mapping algorithm is applied to each element $\Omega$ (with spherical area $\Delta \Omega$) so without loss of generality consider one element. Let $\Delta A^{(pg)}_k$ and $\Delta A^{(nc)}_\ell$ be the spherical area of the physics grid cell $A^{(pg)}_k$ and CSLAM control volume $A^{(nc)}_\ell$, respectively. The physics grid cells and CSLAM cells respectively span the element without gaps or overlaps
\begin{eqnarray}
\cup_{k=1}^{pg^2}A^{(pg)}_k=\Omega \text{ and } A^{(pg)}_k \cap A^{(pg)}_\ell = \emptyset \quad \forall k\ne \ell,\\
\cup_{k=1}^{nc^2}A^{(nc)}_k=\Omega \text{ and } A^{(nc)}_k \cap A^{(nc)}_\ell = \emptyset \quad \forall k\ne \ell.
\end{eqnarray}
The overlap areas between the $k$-th physics grid cell and CSLAM cells is denoted
\begin{equation}
A_{k\ell}=A^{(pg)}_k \cap A^{(nc)}_\ell,
\end{equation}
so that
\begin{equation}
A^{(pg)}_k=\cup_{l=1}^{nc^2}A_{k\ell}.
\end{equation}
This overlap grid is also referred to as an exchange grid.
\subsection{Mapping tracers from CSLAM to $pg$}\label{sec:nctopg}
For mapping tracer state from the CSLAM grid to any physics grid can be done using exising CSLAM technology, i.e. do a high-order shape-preserving reconstruction of mixing ratio and dry air mass inside each CSLAM control volume and integrate those reconstruction functions over the overlap areas \citep{LNU2010JCP,NL2010JCP}. This algorithm retains the properties of CSLAM: inherent mass-conservation, mixing ratio shape-preservation and linear-correlation preservation. 

In mathematical terms, the dry pressure level thickness integrated over the $k$'th physics grid cell is given by
\begin{equation}
\overline{\Delta p}^{(pg)}_k=\sum_{\ell=1}^{nc^2}\langle \Delta p\rangle_{k\ell},
\end{equation}
where $\langle \Delta p\rangle_{k\ell}$ is the integral over the high-order reconstruction of $\Delta p$ over the overlap area $A_{k\ell}$ divided by the area of the overlap
\begin{equation}
\langle \Delta p\rangle_{k\ell}=\frac{1}{\delta A_{k\ell}}\int_{A_{k\ell}}\left[ \sum_{i+j\le 2}C^{(ij)}_\ell x^{i}y^{j}\right] dA,
\end{equation}
where the reconstruction coefficients $C^{(ij)}_\ell$ in CSLAM cell $\ell$ are computed from the cell average pressure level thicknesses on the CSLAM grid $\Delta p^{(nc)}$ and the numerical integration over overlap areas is done by line-integral quadrature. The details are given in \cite{LNU2010JCP} and not repeated here.

The tracer mass per unit area on the physics grid is given by
\begin{equation}
\overline{m\Delta p}^{(pg)}_k=\sum_{\ell=1}^{nc^2}\langle m\Delta p\rangle_{k\ell},
\end{equation}
where $\langle m\Delta p\rangle_{k\ell}$ is the integral over the high-order reconstruction of $\Delta p$ and $m$ combined using the approach outlined in Appendix B of \cite{NL2010JCP} over the overlap area $A_{k\ell}$
\begin{equation}
\langle m\Delta p\rangle_{k\ell}=\frac{1}{\delta A_{k\ell}}\int_{A_{k\ell}}\left[ \overline{\Delta p}_\ell^{(nc)}\sum_{i+j\le 2}c^{(ij)}_\ell x^{i}y^{j}+{\overline{m}}_\ell^{(nc)}\sum_{i+j\le 2}{\tilde{C}}^{(ij)}_\ell x^{i}y^{j}\right] dA,
\end{equation}
where ${\tilde{C}}^{(00)}_\ell=C^{(00)}_\ell-\overline{\Delta p}^{(nc)}_\ell$ and ${\tilde{C}}^{(ij)}_\ell=C^{(ij)}_\ell$ for $i,j>0$. The mixing ratio on the physics grid is then
\begin{equation}
\overline{m}^{(pg)}_k=\frac{\overline{m\Delta p}^{(pg)}_k}{\overline{\Delta p}^{(pg)}_k}.
\end{equation}



The tendencies from the parameterizations are computed on the physics grid. The tracer tendency in physics grid cell $k$ is denoted $f_k^{(pg)}$. The problem is how to map $f_k^{(pg)}$ to the CSLAM control volumes $f^{(nc)}$ satisfying the following constraints:
\begin{enumerate}
\item {\bf{Local mass-conservation}}
\begin{equation}
f_k^{(pg)}\Delta p^{(pg)}_k=\cup_{\ell=1}^{nc^2}\Delta A_{k\ell}\Delta p^{(nc)}_\ell f^{(nc)}_\ell,
\end{equation}
where $\Delta p^{(pg)}_k$ is the pressure level thickness in physics grid cell $k$ and similarly for $\Delta p^{(nc)}$.
\item {\bf{Shape-preservation in mixing ratio}}: The tendencies mapped to the CSLAM grid should not produce values smaller than the updated physics grid mixing ratios, $m^{(pg)}_k+\Delta tf_k^{(pg)}$, or values smaller than the existing CSLAM mixing ratios [revise for whatever code works best]
\begin{equation}
m^{(min)}=\min \left( m^{(pg)}_k+\Delta t f_k^{(pg)},\left\{ m^{(nc)}_{\ell} |\ell=1,nc^2\right\} \right),
\end{equation}
where $\Delta t$ is the physics time-step. Similarly for maxima
\begin{equation}
m_k^{(max)}=\max \left( m^{(pg)}_k+\Delta t f_k^{(pg)},\left\{ m^{(nc)}_{k\ell} |\ell=1,nc^2\right\} \right),
\end{equation}
\item {\bf{Linear correlation preservation}}: The physics forcing must not disrupt linear tracer correlation between species on the CSLAM grid \citep[see, e.g., ][]{LT2011QJR}.
\item {\bf{Consistency}}: A constant mixing ratio tendency, $cnst$, on the physics grid, $f_k^{(pg)}=cnst$ $\forall k$, must result in the same (constant) forcing on the CSLAM grid, $f_\ell^{(nc)}=f_k^{(pg)}=cnst$ $\forall \ell$.
\end{enumerate}
To motivate the algorithm that will simultaneously satisfy 1-4 it is informative to discuss how `standard' mapping algorithms will violate one or more of the constraints:
\begin{itemize}
\item Conservative remapping: Assume that one remaps the mass-tendencies in exactly the same way as the mapping of mixing ratio state from the CSLAM grid to the physics grid described in section \ref{sec:nctopg}. That is, replace $m$ with $f$ and map from physics grid to the CSLAM grid instead of the other way around. The mapped mass-tendency is $\overline{f\Delta p}^{(pg)}_k$ and due to the properties of the mapping algorithm the mass-tendency is conserved, linear correlation between mass-tendencies are conserved and shape in mass-tendency is preserved.

That said, this approach is problematic. The issue is that the dry pressure level thickness mapped from $pg$ to $nc$, call it $\widetilde{\overline{\Delta p}}^{(nc)}$, differs from $\overline{\Delta p}^{(nc)}$. During physics-dynamics coupling the dry pressure level thickness should remain constant. So when converting the mass-tendencies to mixing ratio tendencies through, e.g.,
\begin{equation}
\overline{m}^{(pg)}_k=\frac{\overline{f\Delta p}^{(pg)}_k}{\overline{\Delta p}^{(pg)}_k},
\end{equation}
a constant mixing ratio tendency is not conserved since $\widetilde{\overline{\Delta p}}^{(pg)}_k\ne {\overline{\Delta p}}^{(pg)}_k$. Basically the constant tendency will map to a field representing the spurious discrepancy between $\widetilde{\overline{\Delta p}}^{(pg)}_k$ and ${\overline{\Delta p}}^{(pg)}_k$. A constant tendency can be preserved by using
\begin{equation}
\overline{m}^{(pg)}_k=\frac{\overline{f\Delta p}^{(pg)}_k}{\widetilde{\overline{\Delta p}}^{(pg)}_k},
\end{equation}
instead, but now mass-conservation is lost since $\widetilde{\overline{\Delta p}}^{(pg)}_k\ne {\overline{\Delta p}}^{(pg)}_k$. This issue is similar to the mass-wind inconsistency found in specified dynamics applications \citep[e.g.][]{JKLSBCRE2001QJR}. 

\begin{figure}[t]
\begin{center}
\noindent\includegraphics[width=30pc,angle=0]{figs/area-schematic.png}\\
\end{center}
\caption{Indice notation for (a) the $pg2$ grid, (b) the $pg3$ grid and (c) their exchange grid. {\color{red}Peter - do you think you will use this figure?}}
\label{fig:area-schematic}
\end{figure}

\begin{figure}[t]
\begin{center}
\noindent\includegraphics[width=30pc,angle=0]{figs/alg-schematic.png}\\
\end{center}
\caption{Make captions stand-alone while being concise}
\label{fig:alg-schematic}
\end{figure}

Even if one could derive a reversible map for mapping $\Delta p$ from physics grid to the CSLAM grid there could still be problems with driving mixing ratios negative on the CSLAM grid (we refer to this as the `negativity problem'). This problem is depicted schematically in Figure~\ref{fig:alg-schematic}. Consider a single element of CSLAM control volumes, containing only a single cell with mixing ratio $1.0$, and $0.0$ everywhere else ($m_l$; Figure ~\ref{fig:alg-schematic}a). Assume that the mixing ratios mapped to the $pg2$ grid ($m_k$; Figure~\ref{fig:alg-schematic}b) results in a negative tracer tendency from the physics ($f_k$; Figure~\ref{fig:alg-schematic}c). The non-zero values of the tendencies for $pg2$ areas overlapping CSLAM grid cells originally containing a mixing ratio of zero ($f_{k,l}$; Figure~\ref{fig:alg-schematic}d), are driven negative by the mapped tendency (Figure~\ref{fig:alg-schematic}e). 

%In the $pg2$ configuration, mapping the fields to and from the quadrature grid and $pg2$ grid is identical to that described in H18. As discussed above above, in mapping to the physics grid, CAM-SE's Lagrange basis functions are integrated over the $pg2$ control volumes to provide the physics with a volume averaged state. The procedure is accurate to machine precision, conserves thermal energy and dry air mass, and is consistent (i.e., the mapping preserves a constant). The reverse mapping, from the physics grid to the quadrature grid, is done using a tensor-product Lagrange interpolation (see Appendix A in H18). The Lagrange interpolation is consistent, conserves dry air mass ({\color{red}{Peter, is this true?}}), but does not conserve thermal energy. Errors arising from the lack of energy conservation were estimated to be small; about two orders of magnitude less than the energy dissipation due to the dynamical core alone.

%The semi-Lagrangian advection of tracers in our $pg2$ configuration is solved on the CSLAM grid. 





\item Interpolation: Traditional Lagrange interpolate of the mixing ratio tendency would preserve a constant and could be made shape-preserving using {\em{ad hoc}} filters \citep[e.g.][]{BC2002MWR} but will not inherently preserve mass tendency and suffers from the `negativity problem' described above.
\end{itemize}
As illustrated above none of the standard interpolation or remapping methods will simultaneously satisfy 1-4.
\subsection{Algorithm}
{\color{red}{mention that the reason we map tendency and not state is to avoid spurious tendencies solely due to interpolation errors, i.e. zero tendency on physics grid would transform into tendencies on the CSLAM grid.}}
Define the $\Delta m_{k\ell}$ is the amount of mixing ratio that can be removed without producing new extrema in $m_{k\ell}$
\begin{equation}
\Delta m_{k\ell}=\overline{m}_{k\ell}-m^{(min)},
\end{equation}
where $\overline{m}_{k\ell}$ has been computed using higher-order mapping ....
{\color{red}{mention why the problem is well-posed}}


\section{Results}

\section{Conclusions}


%Text here ===>>>

%%

%  Numbered lines in equations:
%  To add line numbers to lines in equations,
%  \begin{linenomath*}
%  \begin{equation}
%  \end{equation}
%  \end{linenomath*}

%% Enter Figures and Tables near as possible to where they are first mentioned:
%
% DO NOT USE \psfrag or \subfigure commands.
%
% Figure captions go below the figure.
% Table titles go above tables;  other caption information
%  should be placed in last line of the table, using
% \multicolumn2l{$^a$ This is a table note.}
%
%----------------
% EXAMPLE FIGURE
%
% \begin{figure}[h]
% \centering
% when using pdflatex, use pdf file:
% \includegraphics[width=20pc]{figsamp.pdf}
%
% when using dvips, use .eps file:
% \includegraphics[width=20pc]{figsamp.eps}
%
% \caption{Short caption}
% \label{figone}
%  \end{figure}
%
% ---------------
% EXAMPLE TABLE
%
% \begin{table}
% \caption{Time of the Transition Between Phase 1 and Phase 2$^{a}$}
% \centering
% \begin{tabular}{l c}
% \hline
%  Run  & Time (min)  \\
% \hline
%   $l1$  & 260   \\
%   $l2$  & 300   \\
%   $l3$  & 340   \\
%   $h1$  & 270   \\
%   $h2$  & 250   \\
%   $h3$  & 380   \\
%   $r1$  & 370   \\
%   $r2$  & 390   \\
% \hline
% \multicolumn{2}{l}{$^{a}$Footnote text here.}
% \end{tabular}
% \end{table}

%% SIDEWAYS FIGURE and TABLE 
% AGU prefers the use of {sidewaystable} over {landscapetable} as it causes fewer problems.
%
% \begin{sidewaysfigure}
% \includegraphics[width=20pc]{figsamp}
% \caption{caption here}
% \label{newfig}
% \end{sidewaysfigure}
% 
%  \begin{sidewaystable}
%  \caption{Caption here}
% \label{tab:signif_gap_clos}
%  \begin{tabular}{ccc}
% one&two&three\\
% four&five&six
%  \end{tabular}
%  \end{sidewaystable}

%% If using numbered lines, please surround equations with \begin{linenomath*}...\end{linenomath*}
%\begin{linenomath*}
%\begin{equation}
%y|{f} \sim g(m, \sigma),
%\end{equation}
%\end{linenomath*}

%%% End of body of article

%%%%%%%%%%%%%%%%%%%%%%%%%%%%%%%%
%% Optional Appendix goes here
%
% The \appendix command resets counters and redefines section heads
%
% After typing \appendix
%
%\section{Here Is Appendix Title}
% will show
% A: Here Is Appendix Title
%
%\appendix

%\section{Here is a sample appendix}

%%%%%%%%%%%%%%%%%%%%%%%%%%%%%%%%%%%%%%%%%%%%%%%%%%%%%%%%%%%%%%%%
%
% Optional Glossary, Notation or Acronym section goes here:
%
%%%%%%%%%%%%%%  
% Glossary is only allowed in Reviews of Geophysics
%  \begin{glossary}
%  \term{Term}
%   Term Definition here
%  \term{Term}
%   Term Definition here
%  \term{Term}
%   Term Definition here
%  \end{glossary}

%
%%%%%%%%%%%%%%
% Acronyms
%   \begin{acronyms}
%   \acro{Acronym}
%   Definition here
%   \acro{EMOS}
%   Ensemble model output statistics 
%   \acro{ECMWF}
%   Centre for Medium-Range Weather Forecasts
%   \end{acronyms}

%
%%%%%%%%%%%%%%
% Notation 
%   \begin{notation}
%   \notation{$a+b$} Notation Definition here
%   \notation{$e=mc^2$} 
%   Equation in German-born physicist Albert Einstein's theory of special
%  relativity that showed that the increased relativistic mass ($m$) of a
%  body comes from the energy of motion of the body—that is, its kinetic
%  energy ($E$)—divided by the speed of light squared ($c^2$).
%   \end{notation}




%%%%%%%%%%%%%%%%%%%%%%%%%%%%%%%%%%%%%%%%%%%%%%%%%%%%%%%%%%%%%%%%
%
%  ACKNOWLEDGMENTS
%
% The acknowledgments must list:
%
% •	All funding sources related to this work from all authors
%
% •	Any real or perceived financial conflicts of interests for any
%	author
%
% •	Other affiliations for any author that may be perceived as
% 	having a conflict of interest with respect to the results of this
% 	paper.
%
% •	A statement that indicates to the reader where the data
% 	supporting the conclusions can be obtained (for example, in the
% 	references, tables, supporting information, and other databases).
%
% It is also the appropriate place to thank colleagues and other contributors. 
% AGU does not normally allow dedications.

%\acknowledgments

%% ------------------------------------------------------------------------ %%
%% Citations

% Please use ONLY \citet and \citep for reference citations.
% DO NOT use other cite commands (e.g., \cite, \citeyear, \nocite, \citealp, etc.).


%% Example \citet and \citep:
%  ...as shown by \citet{Boug10}, \citet{Buiz07}, \citet{Fra10},
%  \citet{Ghel00}, and \citet{Leit74}. 

%  ...as shown by \citep{Boug10}, \citep{Buiz07}, \citep{Fra10},
%  \citep{Ghel00, Leit74}. 

%  ...has been shown \citep [e.g.,][]{Boug10,Buiz07,Fra10}.



%%  REFERENCE LIST AND TEXT CITATIONS
%
% Either type in your references using
%
% \begin{thebibliography}{}
% \bibitem[{\textit{Kobayashi et~al.}}(2003)]{R2013} Kobayashi, T.,
% Tran, A.~H., Nishijo, H., Ono, T., and Matsumoto, G.  (2003).
% Contribution of hippocampal place cell activity to learning and
% formation of goal-directed navigation in rats. \textit{Neuroscience}
% 117, 1025--1035.
%
% \bibitem{}
% Text
% \end{thebibliography}
%
\bibliography{bib}
%%%%%%%%%%%%%%%%%%%%%%%%%%%%%%%%%%%%%%%%%%%%%%%
% Or, to use BibTeX:
%
% Follow these steps
%
% 1. Type in \bibliography{<name of your .bib file>} 
%    Run LaTeX on your LaTeX file.
%
% 2. Run BiBTeX on your LaTeX file.
%
% 3. Open the new .bbl file containing the reference list and
%   copy all the contents into your LaTeX file here.
%
% 4. Run LaTeX on your new file which will produce the citations.
%
% AGU does not want a .bib or a .bbl file. Please copy in the contents of your .bbl file here.


%% After you run BibTeX, Copy in the contents of the .bbl file here:


%%%%%%%%%%%%%%%%%%%%%%%%%%%%%%%%%%%%%%%%%%%%%%%%%%%%%%%%%%%%%%%%%%%%%
% Track Changes:
% To add words, \added{<word added>}
% To delete words, \deleted{<word deleted>}
% To replace words, \replace{<word to be replaced>}{<replacement word>}
% To explain why change was made: \explain{<explanation>} This will put
% a comment into the right margin.

%%%%%%%%%%%%%%%%%%%%%%%%%%%%%%%%%%%%%%%%%%%%%%%%%%%%%%%%%%%%%%%%%%%%%
% At the end of the document, use \listofchanges, which will list the
% changes and the page and line number where the change was made.

% When final version, \listofchanges will not produce anything,
% \added{<word or words>} word will be printed, \deleted{<word or words} will take away the word,
% \replaced{<delete this word>}{<replace with this word>} will print only the replacement word.
%  In the final version, \explain will not print anything.
%%%%%%%%%%%%%%%%%%%%%%%%%%%%%%%%%%%%%%%%%%%%%%%%%%%%%%%%%%%%%%%%%%%%%

%%%
\listofchanges
%%%

\end{document}

%%%%%%%%%%%%%%%%%%%%%%%%%%%%%%%%%%%%%
%% Supporting Information
%% (Optional) See AGUSuppInfoSamp.tex/pdf for requirements 
%% for Supporting Information.
%%%%%%%%%%%%%%%%%%%%%%%%%%%%%%%%%%%%%



%%%%%%%%%%%%%%%%%%%%%%%%%%%%%%%%%%%%%%%%%%%%%%%%%%%%%%%%%%%%%%%

More Information and Advice:

%% ------------------------------------------------------------------------ %%
%
%  SECTION HEADS
%
%% ------------------------------------------------------------------------ %%

% Capitalize the first letter of each word (except for
% prepositions, conjunctions, and articles that are
% three or fewer letters).

% AGU follows standard outline style; therefore, there cannot be a section 1 without
% a section 2, or a section 2.3.1 without a section 2.3.2.
% Please make sure your section numbers are balanced.
% ---------------
% Level 1 head
%
% Use the \section{} command to identify level 1 heads;
% type the appropriate head wording between the curly
% brackets, as shown below.
%
%An example:
%\section{Level 1 Head: Introduction}
%
% ---------------
% Level 2 head
%
% Use the \subsection{} command to identify level 2 heads.
%An example:
%\subsection{Level 2 Head}
%
% ---------------
% Level 3 head
%
% Use the \subsubsection{} command to identify level 3 heads
%An example:
%\subsubsection{Level 3 Head}
%
%---------------
% Level 4 head
%
% Use the \subsubsubsection{} command to identify level 3 heads
% An example:
%\subsubsubsection{Level 4 Head} An example.
%
%% ------------------------------------------------------------------------ %%
%
%  IN-TEXT LISTS
%
%% ------------------------------------------------------------------------ %%
%
% Do not use bulleted lists; enumerated lists are okay.
% \begin{enumerate}
% \item
% \item
% \item
% \end{enumerate}
%
%% ------------------------------------------------------------------------ %%
%
%  EQUATIONS
%
%% ------------------------------------------------------------------------ %%

% Single-line equations are centered.
% Equation arrays will appear left-aligned.

Math coded inside display math mode \[ ...\]
 will not be numbered, e.g.,:
 \[ x^2=y^2 + z^2\]

 Math coded inside \begin{equation} and \end{equation} will
 be automatically numbered, e.g.,:
 \begin{equation}
 x^2=y^2 + z^2
 \end{equation}


% To create multiline equations, use the
% \begin{eqnarray} and \end{eqnarray} environment
% as demonstrated below.
\begin{eqnarray}
  x_{1} & = & (x - x_{0}) \cos \Theta \nonumber \\
        && + (y - y_{0}) \sin \Theta  \nonumber \\
  y_{1} & = & -(x - x_{0}) \sin \Theta \nonumber \\
        && + (y - y_{0}) \cos \Theta.
\end{eqnarray}

%If you don't want an equation number, use the star form:
%\begin{eqnarray*}...\end{eqnarray*}

% Break each line at a sign of operation
% (+, -, etc.) if possible, with the sign of operation
% on the new line.

% Indent second and subsequent lines to align with
% the first character following the equal sign on the
% first line.

% Use an \hspace{} command to insert horizontal space
% into your equation if necessary. Place an appropriate
% unit of measure between the curly braces, e.g.
% \hspace{1in}; you may have to experiment to achieve
% the correct amount of space.


%% ------------------------------------------------------------------------ %%
%
%  EQUATION NUMBERING: COUNTER
%
%% ------------------------------------------------------------------------ %%

% You may change equation numbering by resetting
% the equation counter or by explicitly numbering
% an equation.

% To explicitly number an equation, type \eqnum{}
% (with the desired number between the brackets)
% after the \begin{equation} or \begin{eqnarray}
% command.  The \eqnum{} command will affect only
% the equation it appears with; LaTeX will number
% any equations appearing later in the manuscript
% according to the equation counter.
%

% If you have a multiline equation that needs only
% one equation number, use a \nonumber command in
% front of the double backslashes (\\) as shown in
% the multiline equation above.

% If you are using line numbers, remember to surround
% equations with \begin{linenomath*}...\end{linenomath*}

%  To add line numbers to lines in equations:
%  \begin{linenomath*}
%  \begin{equation}
%  \end{equation}
%  \end{linenomath*}



